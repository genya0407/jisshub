\documentclass[11pt]{jsarticle}

\usepackage[dvipdfmx]{graphicx}
\usepackage{comment}
\usepackage{listings, jlisting}
\usepackage{amsmath, amssymb}
\usepackage{wrapfig}
\usepackage{fancybox}
\usepackage{ascmac}

\title{電気電子工学実習B シリーズレギュレータ}
\author{三軒家 佑將}
\date{}

\setlength\floatsep{5pt}
\setlength\textfloatsep{5pt}
\setlength\intextsep{5pt}
\setlength\abovecaptionskip{0pt}

\newcommand{\fg}[3]{ % \fg{label}{path}{caption},
    \begin{figure}
        \begin{center}
            \includegraphics[width=\textwidth]{./figs/#2}
            \caption{#3}
            \label{#1}
        \end{center}
    \end{figure}
}

\newcommand{\dummyfig}[2]{
    \begin{figure}
        \begin{center}
            \begin{shadebox}
                手書きの図のため、巻末に添付
            \end{shadebox}
            \caption{#2}
            \label{#1}
        \end{center}
    \end{figure}
}

\newcommand{\cir}[3]{ % \fg{label}{path}{caption},
    \begin{wrapfigure}{R}{0.45\textwidth}
        \begin{center}
            \includegraphics[width=0.4\textwidth]{./figs/#2}
            \caption{#3}
            \label{#1}
         \end{center}
    \end{wrapfigure}
}

\newcommand{\tab}[3]{ % \tab{label}{csv-path}{caption}
    \begin{table}[htb]
          \begin{center}
            \csvautotabular{./data/#2}
            \caption{#3}
            \label{#1}
          \end{center}
    \end{table}
}

\newcommand{\fr}[1]{図\ref{#1}}
\newcommand{\tr}[1]{表\ref{#1}}
\newcommand{\er}[1]{式(\ref{#1})}

\begin{document}
\maketitle

\section{目的}
直流安定化電源を設計・作成し、その動作特性を評価する。

\section{原理・設計}
\fg{fig1}{block.png}{直流安定化電源回路の構成図}
直流安定化電源回路は、交流入力から安定化直流出力を生み出す回路である。直流安定化電源回路は、\fr{fig1}のように構成される。
整流回路は、交流入力を直流出力に変換する回路である。
平滑回路は、整流回路から得られる直流出力から交流成分を取り除き、より直流らしい出力を得る回路である。
安定化電源回路は、リップルや外乱による出力の変動を小さくするための回路である。

% 実験3のRの決め方を解説する

\section{方法}
\subsection{整流回路と平滑回路の実験}
\cir{fig2}{bridge.png}{ブリッジ回路}
\cir{fig3}{filter.png}{コンデンサ入力平滑回路}
\fr{fig2}に示したブリッジ回路を組み立てた。
交流入力電圧が15V(実効値)になるようにスライドレギュレーターを調整した。

この状態で負荷電流を0から230mAまで変化させ、負荷電流に対する負荷電圧特性をグラフに描いた。また、同様に負荷電流に対するリップル含有率の関係をグラフに描いた。また、負荷電流最大のときと最小のときのそれぞれについて、オシロスコープで出力波形を観測し、方眼紙に写生した。

さらに、ブリッジ回路の後段に、\fr{fig3}に示すコンデンサ入力型平滑回路を接続し、上記と同様の実験を行った。

\subsection{定電流回路の実験}
\cir{fig4}{stable-i.png}{定電流回路}
\fr{fig4}の定電流回路を組み立て、入力に直流可変定電圧電源を接続した。
入力電圧$E_i$を0から20Vまで変化させて、出力電流を測定した。
負荷抵抗を変化させ、負荷抵抗の値ごとに入力電圧に対する出力電流の特性曲線を描いた。

さらに、直流可変定電圧電源の代わりにブリッジ整流コンデンサ入力平滑回路を定電流回路に接続し、交流入力電圧を15Vとした。
負荷抵抗を470$\Omega$から3.3${\rm k}\Omega$まで変化させ、出力電流及びリップルを測定し、負荷抵抗に対する出力電流、リップル含有率の特性曲線を描いた。

\subsection{定電圧回路の実験}
\cir{fig5}{stable-v.png}{定電圧回路}
\fr{fig5}の定電圧回路を組み立て、入力に直流可変定電圧電源を接続した。
負荷電流が一定の条件下(0mA, 75mA, 150mA)で、入力電圧${\rm E_i}$を10Vから18Vまで変化させて、出力電圧${\rm E_o}$、基準電圧${\rm V_z}$を測定した。
そして、負荷電流をパラメータとして入力電圧に対する出力電圧、基準電圧の特性曲線を描いた。

さらに、ブリッジ整流コンデンサ入力平滑回路を定電圧回路に接続し、交流入力電圧$E_S$を15Vとした。
負荷電流を0から200mAまで20mA間隔で変化させ、出力電圧及びリップルを測定し、負荷電流に対する
出力電圧、リップル含有率の特性曲線を描いた。
また、出力電圧、ツェナーダイオードの両端の波形を観測し、方眼紙に写した。

また、ブリッジ整流コンデンサ入力平滑回路を接続した回路において、負荷電流を一定(0mA, 75mA, 150mA)として、スライドレギュレータにより交流入力電圧$E_S$を15V$\pm$2Vの範囲で変化させて出力電圧を測定し、入力電圧変動に対する出力電圧の特性曲線を描いた。

\section{実験結果}
\subsection{整流回路と平滑回路の実験}
\fg{fig6}{{1.voltage}.png}{負荷電流ー負荷電圧特性}
\fg{fig7}{{1.ripple}.png}{負荷電流ーリップル含有率特性}
\dummyfig{fig8}{負荷電流最大時の出力電圧波形}
\dummyfig{fig9}{負荷電流最小時の出力電圧波形}

\fr{fig6}はブリッジ整流回路の負荷電流ー負荷電圧特性をグラフに描いたものである。負荷電流が増加するに連れて、負荷電圧が減少していることがわかる。
\fr{fig7}はブリッジ整流回路の負荷電流ーリップル含有率特性をグラフに描いたものである。初めは負荷電流が増加するに連れてリップル含有率が増加するが、負荷電流が50mAを超えたあたりからはリップル含有率が大きく変化しなくなっていることがわかる。
\fr{fig8}、\fr{fig9}はそれぞれ、負荷電流が最大のときと最小のときのブリッジ整流回路の出力電圧の波形をスケッチしたものである。
どちらの波形も、正弦波の負の部分を反転したような形をしているのがわかる。また、負荷電流最大時の波形に比べて、最小時の波形の方が山が高くなっているように見える。

\fg{fig10}{{2.voltage}.png}{負荷電流ー負荷電圧特性}
\fg{fig11}{{2.ripple}.png}{負荷電流ーリップル含有率特性}
\dummyfig{fig12}{負荷電流最大時の出力電圧波形}
\dummyfig{fig13}{負荷電流最小時の出力電圧波形}

\fr{fig10}はブリッジ整流コンデンサ入力平滑回路の負荷電流ー負荷電圧特性をグラフに描いたものである。負荷電流が増加するに連れて、負荷電圧が減少していることがわかる。
また、ブリッジ整流回路と比べると、同じ負荷電流が流れているときでも、負荷電圧が高く、また、負荷電流の変化による負荷電圧の変化が大きい。
\fr{fig11}はブリッジ整流コンデンサ入力平滑回路の負荷電流ーリップル含有率特性をグラフに描いたものである。負荷電流が増加するに連れてリップル含有率が増加していることがわかる。
また、ブリッジ整流回路においては負荷電流が0mAのときもリップル含有率が1.6\%程度あったのに対して、負荷電流が0mAのときはリップル含有率はほぼ0\%であり、全体としても、リップル含有率はブリッジ整流回路に比べて低い。
\fr{fig12}、\fr{fig13}はそれぞれ、負荷電流が最大のときと最小のときのブリッジ整流コンデンサ入力平滑回路の出力電圧の波形をスケッチしたものである。
負荷電流が大きいときは、出力波形にリップルがはっきりと表れている一方、負荷電流が小さいときは、直流に近い平坦な出力が得られている事がわかる。

\subsection{定電流回路の実験}
\fg{fig14}{{3.voltage}.png}{入力電圧ー出力電流特性}
\fr{fig14}は、定電流回路に入力として直流可変定電圧電源を接続し、負荷抵抗の値をパラメーターにして、定電流回路の入力電圧と負荷電流の関係をグラフに描いたものである。
負荷抵抗が小さいときは、入力電圧が低くても所望の出力電流が得られるが、負荷抵抗が大きいときは、入力電圧を上げないと所望の出力電流が得られないことがわかる。
また、負荷抵抗が、47$\Omega$、470$\Omega$、1000$\Omega$の場合は、入力電圧が一定値以上になると、出力電流は一定となり、それ以上増加しないということがわかる。

\fg{fig15}{{4.current}.png}{負荷抵抗ー出力電流特性}
\fg{fig16}{{4.ripple}.png}{負荷抵抗ーリップル含有率特性}
\dummyfig{fig17}{負荷抵抗が1500$\Omega$のときの出力電圧波形}
\fr{fig15}は、定電流回路に入力としてブリッジ整流コンデンサ入力平滑回路を接続し、負荷抵抗に対する出力電流の特性をグラフを描いたものである。
負荷抵抗が小さいうちは出力電流が一定であるが、負荷抵抗がある値より大きくなると、それ以降は負荷抵抗の増加に伴って出力電流が減少していることがわかる。
\fr{fig16}は、同様に、負荷抵抗に対するリップル含有率の特性をグラフに描いたものである。
こちらも、負荷抵抗が小さいうちはリップル含有率は非常に低い値で一定であるが、負荷抵抗がある値より大きくなると、そこから上昇を始め、さらに、負荷抵抗がある値まで大きくなると、そこからは高い値で再び一定となる事がわかる。
\fr{fig17}は、リップル含有率の変化が進行中である負荷抵抗$=1500\Omega$のときの、出力電圧の波形である。包絡線の上部を切り取ったような形をしている。

\subsection{定電圧回路の実験}
\fg{fig18}{{5.output}.png}{入力電圧ー出力電圧特性}
\fg{fig19}{{5.zener}.png}{入力電圧ー基準電圧特性}
\fr{fig18}は、定電圧回路の入力電圧に対する出力電圧の特性をグラフに描いたものである。
\fr{fig19}は、定電圧回路の入力電圧に対する基準電圧の特性をグラフに描いたものである。

\fg{fig20}{{6.voltage}.png}{負荷電流ー出力電圧特性}
\fg{fig21}{{6.ripple}.png}{負荷電流ーリップル含有率特性}
\dummyfig{fig22-1}{負荷電流が大きいときの出力電圧波形}
\dummyfig{fig22-2}{負荷電流が小さいときの出力電圧波形}
\dummyfig{fig23-1}{負荷電流が小さいときのツェナーダイオードの両端の波形}
\dummyfig{fig23-2}{負荷電流が大きいときのツェナーダイオードの両端の波形}
\fr{fig20}は、定電圧回路にブリッジ整流コンデンサ入力平滑回路を接続し、負荷電流に対する出力電圧の特性をグラフに描いたものである。
負荷電流が少ないときは概ね一定の出力電圧を保っているが、負荷電流がある値を超えると、負荷電流が大きくなるに連れて出力電圧が減少するようになることがわかる。
\fr{fig21}は、定電圧回路にブリッジ整流コンデンサ入力平滑回路を接続し、負荷電流に対するリップル含有率の関係をグラフに描いたものである。
負荷電流が少ないときは小さな値を保っているが、負荷電流がある値を超えると、負荷電流が大きくなるに連れてリップル含有率が上昇するようになることがわかる。
\fr{fig22-1}は、負荷電流が大きいときの出力電圧波形をスケッチしたものである。
包絡線の上部を切り取ったような形をしている。
\fr{fig22-2}は、負荷電流が小さいときの出力電圧波形をスケッチしたものである。
包絡線のような形をしている。

\fr{fig23-1}は、負荷電流が小さいときのツェナーダイオードの両端の波形をスケッチしたものである。
包絡線のような形をしている。
\fr{fig23-2}は、負荷電流が大きいときのツェナーダイオードの両端の波形をスケッチしたものである。
単純な正弦波の絶対値のような形をしているが、各山が左右非対称で、単純な正弦波の絶対値ではないようにみえる。

\fg{fig24}{{7.output}.png}{入力電圧ー出力電圧特性}
\fr{fig24}は、負荷電流をパラメータとし、ブリッジ整流回路への交流入力の電圧に対する出力電圧の特性をグラフに描いたものである。
負荷電流が小さいときは、交流入力の電圧がある程度下がっても、出力電圧は概ね一定を保つことができているが、負荷電流が大きいときは、交流入力の電圧が下がるに連れて、出力電圧も減少していることがわかる。

\section{考察}
\end{document}
