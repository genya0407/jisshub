\documentclass[11pt]{ltjsarticle}

\usepackage{fontspec}
\usepackage{luatexja-fontspec}
\usepackage[hiragino-pron]{luatexja-preset}
\usepackage{luatexja-ruby}
\usepackage{graphicx}
\usepackage{amsmath, amssymb}
\usepackage{csvsimple}

\title{実習B 通信システム設計演習}
\author{三軒家 佑將 \\ 1026-26-5817}
\date{}

\newcommand{\fg}[3]{ % \fg{label}{path}{caption},
	\begin{figure}
		\fbox{
			\includegraphics[width=\textwidth]{./communication_system_2016_12_02/#2}
		}
		\caption{#3}
		\label{#1}
	\end{figure}
}

\newcommand{\tab}[3]{ % \tab{label}{csv-path}{caption}
	\begin{table}[htb]
  		\begin{center}
			\csvautotabular{#2}
    		\caption{#3}
			\label{#1}
  		\end{center}
	\end{table}
}

\newcommand{\fr}[1]{図\ref{#1}}
\newcommand{\tr}[1]{表\ref{#1}}

\begin{document}
\maketitle

\section{目的}
	アナログ無線受信機の3方式、すなわち、ストレート受信機、スーパーヘテロダイン受信機、同期検波受信機について、National Instruments 社のシュミレーションソフト LabVIEW を用いて受信回路を作成し、特性を解析する。
\section{方法}
	\subsection{LabVIEWの使い方}
		教科書の例に習い、OOK信号を出力する回路を作成した。
	\subsection{用いる素子の特性解析}
	\subsection{ラジオ受信機の作成}
	\subsection{ラジオ受信機の特性解析}
		\subsubsection{アナログ信号}
		\subsubsection{デジタル信号}
\section{結果} % プログラムの画像、出力されたグラフ、実験結果の表
	\subsection{LabVIEWの使い方}
		\fg{fg1}{1.how_to_use/how__to__used.png}{OOK信号出力回路}
		\fr{fg1}のように回路を作成した。
	\subsection{用いる素子の特性解析}
	\subsection{ラジオ受信機の作成}
	\subsection{ラジオ受信機の特性解析}
		\subsubsection{アナログ信号}
		\subsubsection{デジタル信号}
\section{考察} % 課題の考察

\end{document}
