\documentclass[11pt]{ltjsarticle}

\usepackage{fontspec}
\usepackage{luatexja-fontspec}
\usepackage[hiragino-pron]{luatexja-preset}
\usepackage{luatexja-ruby}
\usepackage{graphicx}
\usepackage{amsmath, amssymb}

\title{実習B 材料工学}
\author{三軒家 佑將}
\date{}

\newcommand{\fig}[3]{
	\begin{figure}
		\includegraphics[width=\textwidth]{#2}
		\caption{#3}
		\label{#1}
	\end{figure}
}
\newcommand{\fr}[1]{図\ref{#1}}

\begin{document}
\maketitle

\section{実習の目的}
	簡単なSiデバイス(太陽電池)の作製を通じて、Siデバイスの作製方法について理解を深める。また、太陽電池・LEDの動作特性について理解を深める。
\section{バイポーラ集積回路について}
	太陽電池の作製プロセスを図にすると\fr{fig1}のようになる。
	\fig{fig1}{resources/process.pdf}{太陽電池の作製プロセス}
	
	図の説明は以下のようである。
	\begin{description}
		\item[①→②]Si基盤を熱酸化し、酸化膜を作成
		\item[②→③]フォトリソグラフィを行い、図のようにフォトレジストを作成する
		\item[③→④]エッチングを行い、酸化膜を剥がす
		\item[④→⑤]フォトレジストを除去する
		\item[⑤→⑥]上面にリン拡散源を、下面にホウ素拡散源を塗布する
		\item[⑥→⑦]熱拡散を行い、${\rm n^+ - Si}$と${\rm p^+ - Si}$を形成する
		\item[⑦→⑧]${\rm SiO_2}$を除去する
		\item[⑧→⑨]電極を形成
	\end{description}

\section{実験で行った内容、条件}
	\subsection{太陽電池の作製}
		前節のプロセスに従って、太陽電池を作製した。その際の条件について以下に示す。
		
		Si基盤として用いたのは、p型Siウェーハから切断した$12mm \times 12mm$の正方形、厚さ$250 \mu m$の小片である。
		
		熱酸化膜形成については、1100℃・90分のドライ加熱を行った。
		
		フォトリソグラフィに際しては、
		\begin{enumerate}
			\item プリベーク 90℃ 90秒
			\item 露光 40秒
			\item 反転ベーク120℃ 90秒
			\item 露光 180秒
			\item 現像 90秒
			\item ポストベーク 120℃ 30秒
		\end{enumerate}
		の手順で行った。また、スコンピータは4000rpmで30秒動作させ、現像の際のマスクには$8mm \times 8mm$の正方形のものを用いた。
		
		拡散源塗布の際には、拡散源を塗布した後に、120℃ 3分のベークを行った。
		
		熱拡散は、900℃ 3分のベークを行った。
		
		電極形成の際には、手法としては真空蒸着を、電極用の金属としてはAlを用いた。また、表面のマスクには葉脈上のものを、裏面のマスクとしては正方形のものを用いた。

		また、出来上がった太陽電池には、導電性ペーストを用いて、裏面には電子回路基板の金属部分を接着し、表面には細い導線を接着した。

	\subsection{太陽電池の特性測定}
		以下の回路(\fr{fig2},\fr{fig3})を用いて太陽電池の特性を測定した。暗所での特性は\fr{fig2}を用いて、電灯で照らしたときの特性は\fr{fig2}を用いた。
		電灯で照らしたときの特性については、電灯の高さが、5cm, 15cm, 25cmの3つの場合について特性を調べた。
		\fig{fig2}{resources/circuit/dark.png}{暗所の測定回路}
		\fig{fig3}{resources/circuit/light.png}{光照射時の測定回路}

	\subsection{LEDの特性測定}
		\fr{fig2}と同様の回路を用いて、市販の赤・黄・緑・青・白色のLEDの特性を測定した。

\section{実験結果}
	\subsection{太陽電池の作製}
		(割愛)
	\subsection{太陽電池の特性測定}
		\label{sec1}
		班員4名の太陽電池について、それぞれV-I特性をグラフにしたのが\fr{fig4}〜\fr{fig7}である。
		\fig{fig4}{resources/solar/base/sangenya.png}{三軒家の太陽電池の特性}
		\fig{fig5}{resources/solar/base/nagura.png}{名倉の太陽電池の特性}
		\fig{fig6}{resources/solar/base/nakagawa.png}{中川の太陽電池の特性}
		\fig{fig7}{resources/solar/base/itano.png}{板野の太陽電池の特性}

	\subsection{LEDの特性測定}
		五色のLEDについて、V-I特性をグラフにしたのが\fr{fig8}である。
		\fig{fig8}{resources/led/led.png}{LEDの特性}

\section{課題1〜3}
	\subsection{課題1}
		\ref{sec1}の結果を見ると、名倉・三軒家と中川・板野の2つのグループに別れていることがわかる。以下では、2つのグループの代表として、三軒家と中川の太陽電池について考察をすすめる。
		このような2つのグループに分かれてしまったのは、太陽電池の表側の電極が$p^+$領域にはみ出てしまい、ダイオードに平行な抵抗が接続されたような状態になってしまったことが原因と考えられる。
		太陽電池だけの特性を見るためには、グラフの直線部分から並列抵抗の値を求め、そこに流れる電流値を計算して、実験データの電流値から引けば良いと考えられる。
		以上のような操作を中川の太陽電池の特性に対して行うと、\fr{fig9}のようになる。
		\fig{fig9}{resources/solar/diff/nakagawa.png}{中川の太陽電池のダイオード部のみの特性(暗所)}
		これを見ると、他方のグループの暗所での特性とよく似た形状のグラフが得られる事がわかる。
		
		\fr{fig4}の暗所のデータと、\fr{fig9}のデータについて、片対数グラフにプロットすると、\fr{fig10}・\fr{fig11}のようになる。
		\fig{fig10}{resources/solar/logscale/sangenya.png}{三軒家の太陽電池の特性(片対数)}
		\fig{fig11}{resources/solar/logscale/nakagawa.png}{中川の太陽電池の特性(片対数)}
		
		ダイオードのI-V特性は、
		\[
			I = I_0 \left( exp(\frac{eV}{nkT}) - 1 \right)
		\]
		のように表され(e: 電気素量, n: 理想ダイオード因子, k: ボルツマン定数, T: 絶対温度)、Vがある一定の範囲のとき、
		\begin{eqnarray}
			I & \approx & I_0 exp(\frac{eV}{nkT}) \nonumber \\
			\therefore log I & \approx & \frac{e}{nkT}V + log I_0 \label{eq1}
		\end{eqnarray}
		と近似できる。\fr{fig10}・\fr{fig11}の直線部分が、式(\ref{eq1})によって近似できるとすると、それぞれ、
		\begin{description}
			\item[sangenya] $ \frac{e}{nkT} = 10.4, \therefore n = 3.72 $
			\item[nakagawa] $ \frac{e}{nkT} = 8.06, \therefore n = 4.80 $
		\end{description}
		となる。Iが拡散電流と再結合電流のみによると考えると、nは1〜2の値になると考えられるので、この結果と一致しない。
		
		このことの説明としては、Si基板中のトラップに一部のキャリアが捕らわれてしまい、電流に寄与しなくなってしまった、ということが考えられる。
		ただし、\fr{fig10}・\fr{fig11}の直線部は、外れ値はあるにせよ、きれいな直線になっていることから、トラップの効果もVに対する指数関数で表されることが必要と考えられるが、実際にそうであるのかは調べてもわからなかった。
		
	\subsection{課題2}
		
	\subsection{課題3}
\section{発電コンテンストについて}
\section{LEDについて}
\section{全体に関する考察、感想、提案など}

\end{document}