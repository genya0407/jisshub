\documentclass[11pt]{ltjsarticle}

\usepackage{fontspec}
\usepackage{luatexja-fontspec}
\usepackage[hiragino-pron]{luatexja-preset}
\usepackage{luatexja-ruby}
\usepackage{graphicx}

\title{実習B 材料工学}
\author{三軒家 佑將}
\date{}

\newcommand{\fig}[3]{
	\begin{figure}
		\includegraphics[width=\textwidth]{#2}
		\caption{#3}
		\label{#1}
	\end{figure}
}
\newcommand{\fr}[1]{図\ref{#1}}

\begin{document}
\maketitle

\section{実習の目的}
	簡単なSiデバイス(太陽電池)の作成を通じて、Siデバイスの作成方法について理解を深める。また、太陽電池・LEDの動作特性について理解を深める。
\section{バイポーラ集積回路について}
	太陽電池の作成プロセスを図にすると\fr{fig1}のようになる。
	\fig{fig1}{resources/process.pdf}{太陽電池の作成プロセス}
	
	図の説明は以下のようである。
	\begin{description}
		\item[①→②]Si基盤を熱酸化し、酸化膜を作成
		\item[②→③]フォトリソグラフィを行い、図のようにフォトレジストを作成する
		\item[③→④]エッチングを行い、酸化膜を剥がす
		\item[④→⑤]フォトレジストを除去する
		\item[⑤→⑥]上面にリン拡散源を、下面にホウ素拡散源を塗布する
		\item[⑥→⑦]熱拡散を行い、${\rm n^+ - Si}$と${\rm p^+ - Si}$を形成する
		\item[⑦→⑧]${\rm SiO_2}$を除去する
		\item[⑧→⑨]電極を形成
	\end{description}
\section{実験で行った内容、条件}
\section{実験結果}
\section{課題1〜3}
\section{発電コンテンストについて}
\section{LEDについて}
\section{全体に関する考察、感想、提案など}

\end{document}